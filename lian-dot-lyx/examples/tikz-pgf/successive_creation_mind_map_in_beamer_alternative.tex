\documentclass{beamer} 
\RequirePackage{helvet} 
\usepackage{tikz} 
\usetikzlibrary{mindmap,trees,shadows} 

\begin{document} 

\makeatletter 
\def\tikz@collect@child hild{% 
  \pgfutil@ifnextchar<{\tikz@collect@child@overlay}% 
  {\pgfutil@ifnextchar[{\tikz@collect@childA}{\tikz@collect@childA[]}}% 
} 
\def\tikz@collect@child@overlay<#1> 

\def\tikz@collect@child@@overlay#1[{\tikz@collect@childA[child overlay={#1},} 

\def\tikzprocessoverlay#1#2#3{% 
  \def\beamer@doifinframe{#2}% 
  \def\beamer@doifnotinframe{#3}% 
  \beamer@masterdecode{#1}% 
  \beamer@donow% 
} 

% Extra hackery to allow preactions on different layers. 
% 
\def\tikz@extra@preaction#1{% 
  {% 
    \pgfsys@beginscope% 
    \setbox\tikz@figbox=\box\voidb@x% 
    \begingroup\tikzset{#1}\expandafter\endgroup% 
    \expandafter\def\expandafter\tikz@preaction@layer\expandafter{\tikz@preaction@layer}% 
    \ifx\tikz@preaction@layer\pgfutil@empty% 
    \path[#1];% do extra path 
    \else% 
    \begin{pgfonlayer}{\tikz@preaction@layer}% 
      \path[#1];% 
    \end{pgfonlayer} 
    \fi% 
    \pgfsyssoftpath@setcurrentpath\tikz@actions@path% restore 
    \tikz@restorepathsize% 
    \pgfsys@endscope% 
  }% 
} 
\let\tikz@preaction@layer=\pgfutil@empty 

\tikzset{preaction layer/.store in=\tikz@preaction@layer} 

\makeatother 

\tikzset{% 
  child overlay/.code={% 
    \tikzprocessoverlay{#1}{}% 
    {% 
      \tikzset{% 
        circle connection bar switch color/.code={}, 
        edge from parent/.style={draw=none}, 
        every node/.style={ 
          concept, draw=none, fill=none, 
          execute at begin node={\setbox0=\hbox\bgroup\hskip0pt\let\\=\relax}, 
          execute at end node=\egroup\phantom{\box0} 
        }% 
      }% 
    }% 
  } 
} 

\pgfdeclarelayer{shadow} 
\pgfsetlayers{shadow,main} 

\tikzset{ 
  use shadow/.style={% 
    copy shadow={% 
      preaction layer=shadow, fill=gray!25, draw=none, 
      shadow xshift=0.5ex, shadow yshift=-0.5ex 
    } 
  }, 
  small mindmap/.style={ 
    level 1/.append style={level 1 concept}, 
    level 2/.append style={level 2 concept}, 
    level 3/.append style={level 3 concept}, 
    level 4/.append style={level 4 concept}, 
    every concept/.style={align=center, font=\tiny\strut, text=black, 
      outer sep=-.25pt}, 
    text width=2cm, 
    concept color=root color, 
    level 1 concept/.style={ 
      text width=1.5cm, 
      level distance=3cm, 
      sibling angle=75, 
      counterclockwise from=285, 
      every child/.style={concept color=level 1 color}, 
    }, 
    level 2 concept/.style={ 
      text width=1.125cm, 
      level distance=3cm, 
      sibling angle=30, 
      clockwise from=30, 
      every child/.style={concept color=level 2 color},   
    }, 
    level 3 concept/.style={ 
      text width=1cm, 
      level distance=3cm, 
      sibling angle=30, 
      clockwise from=30, 
      every child/.style={concept color=level 3 color}, 
    },  
    every node/.style={concept, execute at begin node=\hskip0pt, use shadow}, 
    every circle connection bar/.append style={append after 
      command={[use shadow]}} 
  }, 
  spoken language/.style=, 
  figurative utterances/.style=, 
  irony/.style=, 
  sarcasm/.style=, 
} 

\colorlet{root color}{blue!25} 
\colorlet{level 1 color}{purple!50} 
\colorlet{level 2 color}{red!50} 
\colorlet{level 3 color}{pink!50} 
\colorlet{highlight}{orange!25!yellow} 

\pgfkeysdefargs{/tikz/assign color}{#1 to #2}{\colorlet{#2}{#1}} 

\begin{frame} 

  \frametitle{Figurative language, irony and sarcasm} 

  \begin{tikzpicture}[remember picture, overlay, small mindmap] 

    \only<14>{ 
      \tikzset{ 
        concept color=highlight, 
        figurative utterances/.style={assign color=highlight to level 1 color}, 
        irony/.style={assign color=highlight to level 2 color}, 
        sarcasm/.style={assign color=highlight to level 3 color}, 
      } 
    } 
    \node <1-> at ([shift={(0cm,0cm)}]current page.west) [spoken language] 
    {Spoken \\ Language} 
    child <2-> {node {Literal \\ Utterances}} 
    child <3->  [figurative utterances] {node {Figurative \\ Utterances} 
      child <4-> {node {Metaphor}} 
      child <5-> {node {Simile}} 
      child <6-> [irony] {node  {Irony} 
        child <9-> {node {Hyperbole}} 
        child <10-> {node {Understatement}} 
        child <11-> {node {Rhetorical Questions}} 
        child <12-> [sarcasm] {node  {Sarcasm}} 
        child <13-> {node {Jocularity}} 
      } 
      child <7->{node {Idiom}} 
      child <8->{node {Indirect \\ Requests}}}; 

  \end{tikzpicture} 

\end{frame} 

\end{document}
