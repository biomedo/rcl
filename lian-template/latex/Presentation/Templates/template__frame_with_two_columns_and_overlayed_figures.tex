% ``columns'' options usage (from the beamer user guide)
%
% [b] will cause the bottom lines of the columns to be vertically aligned.
% [c] will cause the columns to be centered vertically relative to each other. Default, unless the global option t is used.
% [onlytextwidth] is the same as totalwidth=\textwidth.
% [t] will cause the first lines of the columns to be aligned. Default if global option t is used.
% [T] is similar to the t option, but T aligns the tops of the first lines while t aligns the so-called baselines of the first lines. If strange things seem to happen in conjunction with the t option (for example if a graphic suddenly �drops down� with the t option instead of �going up,�), try using this option instead.
% [totalwidth= width] will cause the columns to occupy not the whole page width, but only width, all told.


\begin{frame}
	%
	\frametitle{}
	\framesubtitle{}
	%
	\begin{columns}[c]
		%
		\begin{column}{5cm}
			\begin{overlayarea}{5cm}{3cm} % {width}{height}
				\begin{flushleft}
					%
					aaa \\
					%
					\onslide<1-2 | handout:1>
					{
						\includegraphics<1-2>[height=1cm]{logo_unipd}
					}
					\onslide<3-4 | handout:1>
					{
						\includegraphics<3-4>[height=1cm]{logo_dei__big}
					}
				\end{flushleft}
			\end{overlayarea}
		\end{column}
		%
		\begin{column}{5cm}
			\begin{overlayarea}{5cm}{3cm} % {width}{height}
				\begin{flushright}
					%
					bbb \\
					%
					\onslide<1-1 | handout:0>
					{
						\includegraphics<1-1>[height=1cm]{logo_dei__big}
					}
					\onslide<2-4 | handout:0>
					{
						\includegraphics<2-4>[height=1cm]{logo_unipd}
					}
				\end{flushright}
			\end{overlayarea}
		\end{column}
		%
	\end{columns}
	%
\end{frame}
