\documentclass{article}
\usepackage{my-tex-live-zh-cn}

\begin{document}

\title{XFEM扩展有限元教程}

\author{廉伟东}

\date{2011 年 6 月}

\maketitle

\tableofcontents

\section{简介}
\label{sec:intro}

\begin{enumerate}
\item xFiniteElement这个类的理解,这个类的作用就是输入一
  个mEntity对象,他就能帮你返回对应的xValueKey,记住不管怎么样,不管输
  入的什么样的对象点,线,面,体对象等(dim=0,1,2,3),他都会帮你返
  回点对象的ValueKey,也就是说那些对象都被最终分解为点对象,因为有限元计
  算关心的是节点信息,因此经过这类他就会帮你返回网格中节点对应
  的xValueKey,而通过xValueManager就可以获得对应的xDoubleValue。而整个
  网格对应的xDoubleValue和xValueKey都是通过DeclareInterpolation来创建
  的。
\item xClassRegion这个类可以返回xMesh(mMesh)网格中的点,线,面,体的
  集合,通过这些对象的ntity\_id,注意这个entity\_id,其实就是在gmsh生成
  网格时对应的physical index。
\item xField这个类是连接了xSpace和实际的xDoubleValueManager,因此我们直
  接可以通过xField.beginValues(mEntity)和xField.endValues(mEntity)来返
  回mEntity对应的xDoubleValue,不管mEntity的类型是点线面体。同
  理xField.beginFcts(mEntity)返回对应的shape function。
\item xAssembler类用来集成全局刚度矩阵和全局载荷向量,在该类中分为集成
  矩阵还是向量,若是集成矩阵时要提供I,J,Value信息,如果要集成载荷阵列
  则要提供I,Value信息。并且实际上I,J信息的提供只是需要一个在整体刚度矩
  阵中的位置信息即可,在当前XFEM程序中不采用整型数来编号,而是通过每个
  节点的xDoubleValue的指针,也就是xDoubleValueManager中的Iterator来进
  行。
\item xSpaceLagrange,xSpaceComposite,xSpaceFiltered等对象的主要作用是获
  取形函数(Fcts)和键值的(femKey)。核心工作就是你可以输入一
  个mEntity{*}e,注意这个Entity可以是体,面,线,点各种对象,他会根据定
  义的Space的自由度空间,返回对应节点的ValueKey以及Function。也就是输入
  一个从大的对象返回对应点对象对应的节点自由度信息.具体的实现参见代
  码xSpaceLagrange::getKeys(mEntity{*} e,femKeys{*}keys),其中涉及了对
  各种不同的几何对象如何获取对应几何对象的基本节点,然后再根据空间自由
  度,向量场还是标量场,来生成对应的自由度所对应的xValueKey。所以仅仅是
  生成xValueKey。
\item 当看到xEval是就是重载了括号运算符的一个函数对象,返回值的类型就是
  模板提供的参数。比如xEvalConst表示返回常数,xEvalFctAtPoint返回给定的
  函数在某点的值;xEvalUnary或者xEvalBinary返回指定的单一参数或者二参数
  谓词的返回值,也就是计算指定谓词的返回值。xValOperator and
  xValOperator数值操作运算符。
\item xEval,xEvalConst,xEvalFctAtPoint,xEvalUnary,xEvalBinary,当看到这
  些函数时,要在浅意识有几点认识。第一,只有xEval字样就表示要进行一
  个Evaluation,这个求值,是建立在给定每个单元的xGeoElm m\_appro,
  m\_eteg参数然后返回一个计算的值,计算返回值的类型就是xEval<T>中的T所
  指定的参数;第二,如果看到xEvalUnary and xEvalBinary就意味着首先要对
  在构造函数中传入的两个xEval类型的派生对象进行求值运算,然后他们分别求
  值运算的结果,作为Unary或者Binary的参数,在进行一次Unary或者Binary操
  作符的运算,得到的结果作为xEvalBinary求值运算的返回值。比
  如xEvalBinary< xMult<xTensor4, xTensor2, xTensor2> >
  eval\_stress(hooke, eval\_strain);,这条语句的意思就是说,要进行一次
  求值运算,首先对构造函数传入的参数hook,和eval\_strain分别进行他们对
  应的operator()(xGeoElement{*} appro, xGeoElement{*} integ,
  returntype\& res)进行求值运算,然后他们的返回的计算结果,分别
  是xTensor4和xTensor2类型的,他们分别作为二维谓词xMult的两个参数,再进
  行求值运算,得到的结果作为最终的返回值。因此可以看出整个这
  个xEvalBinary的返回值类型应该是xMult<xTensor4, xTensor2, xTensor2>中
  最后一个参数类型,而xMult<xTensor4,xTensor2,xTensor2>头两个参数的类型
  就分别是hooke和eval\_strain的返回值类型。因此我们可以说xEvalBinary,或
  者xEvalUnary的意思就是首先进行xEval求值运算,然后返回值在进行Binary或
  者Unary操作符对应的运算。
\item 一些概念,对于每个节点都有xValueKey xDoubleValue ShapeFuction,通
  过xfield来管理,xvaluemanager。对于组装整体刚度矩阵,需要对每个单元进
  行积分,求解其单元刚度矩阵,而后装配到整体刚度矩阵。对每个进行积分求
  解单元刚度矩阵,就需要遍历每个单元,每个单元在遍历其中的节点来进行积
  分。
\item xMesh中的xVertex对应的就是节点的几何对象,其中的Id就是gmsh文件中
  的nodeid因此在输出的dcl.dbg文件中也就是xDoubleManager中输出的调试信息
  的Enti id号。这样一旦出现问题时,可以对照gmsh和dcl.dbg文件来找到问
  题。
\item mEntity e\_integ and e\_appro; xGeoElement geo\_integ and
  geo\_appro;区别就是当一个单元被加强时,zerolevel与单元相交,这个时候
  整个单元叫做e\_appro and geo\_appro,而那些小的subelements我们叫
  做e\_Integ and geo\_integ,这个小的单元是用来获取积分点然后进行积分的。
  也就是说对整个大的单元的积分,转变成了对小的单元的积分的和。这样在每
  个小单元内部材料的特性就是一致的。一个大的单元有几个小的sub单元构
  成,但是记住一条规则小的单元的某些在大的单元中间节点上不存
  在shapefunction,因为这些sub-element只是用来积分更加准确,实际上估计
  空间中并不存在这些自由度。这些小的子单元和大的单元享有共同形函数,但
  是他们有自己的积分点和Jacbian行列式。具体的积分点是通
  过xGeoElement::getUVW来获取的。如果有获取常规坐标下的积分点要
  用getXYZ,遍历积分点可以通过for(int k = 0; k < nb;
  k++)geo\_integ->setUVW(k);
\item 如何xspace类型是scalar比如温度,则求得的温度场和场的Gradient操作
  分别是double和xVector类型;如果xspace的类型是Vector比如位移场,则求得
  的位移场以及场的Gradient操作分别为xVector和xTensor2。
\item xFormBiLinear是用来组装左侧的刚度矩阵的,而xFormLinear是用来生成
  对应的载荷阵列的,也就是右侧的向量。如何将双线性形式和线性形式组装到
  对应的矩阵?通过Assemble这个函数就可以将对应的线性和双线形式组装到对
  应的矩阵和向量中。这个函数是重载函数,有多种版本形式.
\item Assemble双线性形式的程序流程如下:首先要确定你的test function的估
  计函数和你的shapefunction是否一致,这里可以指定不同的函数空间,对于常
  规固体力学问题往往采用伽辽金法来建立有限元格式,因此test function的估
  计空间和求解的估计空间是一致的。在程序中通过参数test and trial来进行
  区分,如果选取一致那么就属于加辽金法。首先遍历你输入的积分区域(一般
  可是边,面,体),这样程序实际就是用来处理积分区域中的每个子区域。对
  于每个小的子区域,也就是一条边,一个面,一个体,程序中的变量就
  是e\_integ,另外就是获取估计函数,因为在xfem中需要对一个有两种材料的
  单元进行积分,而这个单元就是通过levelset来分割的,因此我们要对这种单
  元进行分割,然后分别对两侧积分,要注意这只是一种积分策略,实际上我们
  的估计空间中根本不存在这样两个子单元,而只有整个单元,因此对这样的子
  单元积分时就要用到包含这个子单元的大的单元的形函数来积分。这样这个积
  分对象的形函数就不是自身节点对应的形函数了。所以这就是为什么在程序中
  经常看到有e\_integ和e\_appro,分别表示就是积分对象以及这个积分对象的
  估计函数。这样对于每个小的子区域,我们可以通过xFiniteElement这个对象
  获得相应的Shape Functions,这样就获得对应的刚度矩阵了,比如对于一个二
  维三角形单元区域,有六个自由度,如果加上加强节点就有12个自由度,这样
  组合出来对应就是一个12X12的矩阵,矩阵每个元素都对应两个下标,这样你能
  标定把每个元素按照他们在整体刚度矩阵中的位置,组装上去了。要注意两
  点,第一即使是对于一个单元里面的subdomain进行积分,那么形函数个数取决
  于单元区域,而不是取决于subdomain,我们只是从subdomain上获得准去的材
  料特性以及高斯积分点,这样即使是对于这个比自己小的subdomain,那么刚度
  矩阵仍然是12X12大小的。s
\item xFormBiLinear是如何工作的呢?他的工作非常简单就是当你给
  我subdomain对象即e\_integ以及对应的估计函数对象e\_appro,以及对应的形
  函数,他就会对每个高斯积分点处刚度矩阵就行累加,获得最终的subdomain对
  整体刚度矩阵的贡献。主要在xFormBiLinear中,在对每个高斯积分点积分之
  前,它都会把对应的高斯积分点的坐标放置在xGeoElem(e\_appro)中,这样就
  方便我们调用了。但是对于每个高斯积分点如何积分呢,是采用双线性形
  式,但是双线性形式格式是什么样的呢?利用C++语言中的虚拟函数这样我们就
  从xFormBiLinear中实现派生了大量的其它子类,比如xFormBiLinearWithLaw,
  xFormBiLinearWithoutLaw等。
\item xFormBiLInearWithoutlaw中accumulate\_png是首先返回在某个积分点处
  各个形函数对应的值(如何有Operator,就应用那个operator,一般
  是gradient操作符),分别对左项和右侧项执行这个操
  作,叫做values\_left, and values\_right。获取该高斯积分点的权值和对应
  的雅阁比行列式的值wdet,然后将获取的values\_left 12和values\_right
  12,分别相乘并乘以wdet,这样最终就获取了,12X12的刚度矩阵。
\item 关于如何从一个Entity访问与其相关的Entity,比如从节点访问和节点相
  连的单元。mEntity{*} e; e->size(level); e 是一个基类指针,可以指
  向vertex,edge,face,volume等,level表示想获取的对象的size,比如0获
  取当前对象所包含的点的个数,1表示想当前对象包含的线的个数,同理2表示
  面,3表示体;如果当前对象是点,想获取的对象的类型是面,那么这是他就直
  接给你返回与这个点相连的面的个数。

\item 从一个网格我们可以创建一个region,比如xRegion all(data->mesh)。此
  时如果我们利用xPhySurface将这个region分割为matrix和inclusion两个部
  分,要注意此时的tag,我们现在讨论的是all.begin()和all.end()之间的这
  些mEntity,其中分割导致的sub-Entity并不含在其中,主要
  是all.begin()和all.end()之间的实体,也就是说你如果用xRegionFilter
  filter\_regin(all.begin(), all.end(), xAccept("inclusion")),此时获得
  这个时候获取的inclusion部分,不是准确的按照那个levelset边界来划分
  的,因为all.begin()到all.end()中不包含那个sub-Entity。而为什么我们在
  积分的时候就可以呢,因为在积分的时候我们有两个选
  择,xIntergrationRulePartion和xIntergrationRuleBaisc,此时如果你选
  择baisc那么你只是对那个不准确的边界的inclusion部分进行积分,而如果你
  选择用Partition,那么这个规则就会把被边界分割的单元按照几块分别进行积
  分,每小块(i.e. subelement)对应他们自己的材料特性,但是主要此时形函数
  却还是选择那个整块单元所对应的估计函数,这就是为什么你会看
  到e\_appro和e\_entity在xFormBilinear中。因为在xfem中subelement只是为
  了积分策略才存在的,实际上求解空间中并不存在这些子单元对应的形函
  数,所以就是为什么对subelemnt积分时要选择他的父亲单元来获得形函数。
  而subentity和父entity之间的关系都是在xPhysSurface中来分割确定的。
\item xIntegrationRulePartition的构造函数有两个版本,在第二个构造函数
  中我们可以指定一个filter这样的话,我们就可以滤过一些被分割开的
  subelements。这样就可以控制输出结果中是不是包含subelements。

\end{enumerate}

\section{常用代码}
\label{sec:utility-codes}

\begin{enumerate}
\item 填充一个新的xField对象

下面的代码用来fill一个新的field,有的时候经常需要
\definecolor{lbcolor}{rgb}{0.9,0.9,0.9}
\lstset{
    % language=C++,
    keywordstyle=\bfseries\ttfamily\color[rgb]{0,0,1},
    identifierstyle=\ttfamily,
    commentstyle=\color[rgb]{0.133,0.545,0.133},
    stringstyle=\ttfamily\color[rgb]{0.627,0.126,0.941},
    showstringspaces=false,
    basicstyle=\small,
    numberstyle=\footnotesize,
    numbers=left,
    stepnumber=1,
    numbersep=10pt,
    tabsize=2,
    breaklines=true,
    prebreak = \raisebox{0ex}[0ex][0ex]{\ensuremath{\hookleftarrow}},
    breakatwhitespace=false,
    aboveskip={1.5\baselineskip},
    columns=fixed,
    upquote=true,
    extendedchars=true,
    frame=single,
    backgroundcolor=\color{lbcolor}
}

\lstinputlisting[language=C++]{code/fillfield.cc}

\item dfd

\end{enumerate}

\end{document}
