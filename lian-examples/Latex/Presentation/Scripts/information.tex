\def\TITLE			{Presentation template}
\def\SHORTTITLE		{template}
\def\AUTHOR			{Author template}
\def\SHORTAUTHOR	{AT}
\def\INSTITUTE		{Institute template}
\def\SHORTINSTITUTE	{IT}
\def\SUBJECT		{Make a presentation template.}
\def\KEYWORDS		{Presentation, template}
\def\CREATOR		{University of template}
\def\DATE			{March 8 2010}
%
%
\def\THANKS
{
	\thanks
	{
		work supported by Uncle Duck
	}
}
%
\ifpdf \hypersetup
{
	pdftitle		= {\TITLE},
	pdfauthor		= {\AUTHOR},
	pdfsubject		= {\SUBJECT},
	pdfcreator		= {\CREATOR},
	pdfproducer		= {\INSTITUTE},
	pdfkeywords		= {\KEYWORDS},
	%
%	pdfpagemode		= FullScreen,		% se si vuole il full screen quando il documento viene aperto
	pdfstartview	= Fit,				% modalit� di visione del documento aperto (Fit FitH FitV FitR FitB)
	pdfstartpage	= 1,				% pagina su cui si vuole che venga aperto il file
	pdfnewwindow	= true,				% se si vuole che il pdf venga aperto su una finestra nuova
	pdfcenterwindow	= true,				% se si vuole che il pdf venga aperto al centro dello schermo
	pdftoolbar		= false,			% if you want to show Acrobat's toolbar
	pdfmenubar		= false,			% if you want to show Acrobat's menu
	%
	colorlinks		= false,			% false: boxed links    true: colored links
	linkbordercolor	= 0.8 0.8 0.8,		% (RGB) links normali
	citebordercolor	= 0.8 0.8 0.8,		% (RGB) links di citazione
	filebordercolor	= 0.8 0.8 0.8,		% (RGB) links ai files
	urlbordercolor	= 0.8 0.8 0.8,		% (RGB) links agli URL
}
%
\urlstyle{same}							% stile degli URL [same - commentato]
%
\fi
%
\title		[\SHORTTITLE]		{\TITLE}
\author		[\SHORTAUTHOR]		{\AUTHOR}
\date		{\DATE}
\institute	[\SHORTINSTITUTE]	{\INSTITUTE}
%
% if you want the current slide's index
%\author		[\SHORTAUTHOR \\ $\;$ \\ \insertframenumber $\;\!\!$ on \inserttotalframenumber]		{\AUTHOR}


% NOTE: hyperref could cause some warnings like:
%
%    ''Package hyperref Warning: Token not allowed in a PDFDocEncoded string:''
% 
% if you want to remove them you have to substitute the sentence using \texorpdfstring{}{} command like
% in the following example:
% \def\INSTITUTE
% {
%	\texorpdfstring
%	{Department of Information Engineering \\ University of Padova}		% actual TeX string
%	{Department of Information Engineering University of Padova}		% Bookmark used by hyperref
% }

