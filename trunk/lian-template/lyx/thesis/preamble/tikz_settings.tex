% For every picture that defines or uses external nodes, you'll have to
% apply the 'remember picture' style. To avoid some typing, we'll apply
% the style to all pictures.
\tikzstyle{na} = [baseline=-.5ex]
\tikzstyle{every picture}+=[remember picture]


\tikzset
{
  FittingStyle/.style =
  {
    % -- shape properties --
    shape = ellipse,                            % shape
    % rounded corners = 1mm,                  % kind of corner (and radius of the roundness)
    % minimum height  = 0.15\textwidth,       % | minimum size of the node
    % minimum width   = 0.9\textwidth,        % |
    % minimum size    = 4.5mm,                %
    % rotate          = 0,                    % angle of rotation
    scale           = 1.0,                  % scaling factor
    thick,                                  % thickness of the border
    % 
    % -- colours properties --
    % filling: [ trasparent | monocolored | shaded]; decomment what you prefer
    %       %                                       % transparent (all commented)
    % fill            = black!30,             % monocolored
    % top color       = white,                % | filling of the node
    % bottom color    = red!50!black!20,      % |
    % text            = black,                % colour of the fonts
    draw            = red,              % colour of the border
    % 
    % -- fonts --
    % font            = \scriptsize,              % shape of the font (or dimension, like \tiny)
    % text centered,                          % text alignment [text centered | text badly centered | text justified | text ragged | text badly ragged]
    % text height     = 1mm,                  % ! minimum size of the text    % NOT WORKING
    % text depth      = 1mm,                  % !
    inner xsep      = 0mm,                  % minimum distance between text and borders along x dimension
    inner ysep      = 0mm                   % minimum distance between text and borders along y dimension
  }
}

\tikzset
{
  GenericNodeStyle/.style =
  {
    % -- shape properties --
    shape = rectangle,                          % shape
    rounded corners = 3mm,                  % kind of corner (and radius of the roundness)
    minimum height  = 1cm,                  % | minimum size of the node
    minimum width   = 2cm,                  % |
    scale           = 1.0,                  % scaling factor
    thick,                                  % thickness of the border
    % 
    % -- colours properties --
    % filling: [ trasparent | monocolored | shaded]; decomment what you prefer
    %       %                                       % transparent (all commented)
    fill            = green!10!white,               % monocolored
    % top color       = white,                % | filling of the node
    % bottom color    = red!50!black!20,      % |
    % text            = black,                % colour of the fonts
    draw            = green,                % colour of the border
    % 
    % -- fonts --
    % font            = \scriptsize,              % shape of the font (or dimension, like \tiny)
    % text centered,                          % text alignment [text centered | text badly centered | text justified | text ragged | text badly ragged]
    % text height     = 1mm,                  % ! minimum size of the text    % NOT WORKING
    % text depth      = 1mm,                  % !
    inner xsep      = 3mm,                  % minimum distance between text and borders along x dimension
    inner ysep      = 3mm                   % minimum distance between text and borders along y dimension
  }
}




\tikzset
{
  NormalNodeStyle/.style =
  {
    shape = circle,                         % shape
    minimum size    = 20,                   %
    rotate          = 0,                    % angle of rotation
    scale           = 1.0,                  % scaling factor
    thick,                                  % thickness of the border
    text            = black,                % colour of the fonts
    draw            = black,                % colour of the border
    font            = \small,               % shape of the font (or dimension, like \tiny)
    text centered,                          % text alignment
    inner xsep      = 0,                    % minimum distance between text and borders along x dimension
    inner ysep      = 0                     % minimum distance between text and borders along y dimension
  }
}



\tikzset
{
  BridgeNodeStyle/.style =
  {
    circle,                                 % shape
    minimum size    = 20,                   %
    rotate          = 0,                    % angle of rotation
    scale           = 1.0,                  % scaling factor
    thick,                                  % thickness of the border
    fill            = black!30,             % monocolored
    text            = black,                % colour of the fonts
    draw            = black,                % colour of the border
    font            = \small,               % shape of the font (or dimension, like \tiny)
    text centered,                          % text alignment 
    inner xsep      = 0,                    % minimum distance between text and borders along x dimension
    inner ysep      = 0                     % minimum distance between text and borders along y dimension
  }
}






\tikzset
{
  WarningTextStyle/.style =
  {
    rectangle,                      % shape
    rounded corners = 0.6cm,        %
    minimum size    = 2cm,          %
    rotate          = 0,            % angle of rotation
    scale           = 1.0,          % scaling factor
    thick,                          % thickness of the border
    fill            = red!10,       % monocolored
    text            = red!10!black, % colour of the fonts
    draw            = red,          % colour of the border
    % font          = \large,       % shape of the font (or dimension, like \tiny)
    text centered,                  % text alignment 
    text width      = 10cm,         % text alignment 
    inner xsep      = 0.5cm,        % minimum distance between text and borders along x dimension
    inner ysep      = 0.5cm         % minimum distance between text and borders along y dimension
  }
}


\tikzstyle{sNormalImageStyle} =
[
]

\tikzstyle{sSplitTwoPartsStyle} =
[
rectangle split,
rectangle split parts=2,
rectangle split draw splits=false
]

\tikzstyle{sConnectionTextStyle}=
[
midway,
above,
sloped,
align=center,
]

\tikzstyle{sNormalBlockStyle} =
[
draw=blue,
rectangle,
rounded corners = 0.1cm,
fill            = blue!20,
minimum height  = 3em,
minimum width   = 6em,
]

\tikzstyle{sNormalDiamondStyle} =
[
diamond,
draw        = blue,
% thick,
fill        = blue!20,
align       = flush center,
]

\tikzstyle{sWarningBlockStyle} =
[
rectangle,
rounded corners = 0.1cm,
fill            = alerted text.fg!20,       % monocolored
text            = alerted text.fg!20!black, % colour of the fonts
draw            = alerted text.fg,          % colour of the border
minimum height  = 3em,
minimum width   = 6em,
]

\tikzstyle{sFillBlockStyle} =
[
draw = red,
rectangle,
fill            = blue,
fill opacity = 0.4,
line width = 4pt,
minimum height  = 3em,
minimum width   = 6em
]


\tikzstyle{sFillGrayStyle} =
[
% draw = red,
rectangle,
fill            = gray,
fill opacity = 0.5,
line width = 4pt,
rounded corners,
minimum height  = 3em,
minimum width   = 6em
]


\tikzstyle{sSumBlockStyle} =
[
shape           = circle,
draw,
fill            = blue!20,
]

\tikzstyle{sFittingStyle} =
[
shape = rectangle,
draw,
line width = 2pt,
color = red,
inner sep = 2mm,
rounded corners
]

\tikzstyle{sFittingStyleAlt} =
[
shape = rectangle,
draw = red,
fill=none,
line width = 2pt,
inner sep = 2mm,
rounded corners
]

\tikzstyle{sFittingStyleDotted} =
[
sFittingStyle,
dash pattern = on 1pt off 4pt on 6pt off 4pt,
]


\tikzstyle{sArrowsStyleEmph} =
[
line width = 8pt,
color   = red!95!black,
-latex
]


\tikzstyle{sArrowsStyleNormal} =
[
line width = 2pt,
color   = blue,
-latex
]


\tikzstyle{sLinesStyle} =
[
thick,
color   = black,
-
]




\tikzstyle{sTextBlockStyle} =
[
draw,
rectangle,
drop shadow,
rounded corners = 0.1cm,
fill            = blue!10,
thick,
inner xsep      = 0.2cm,        % minimum distance between text and borders along x dimension
inner ysep      = 0.2cm         % minimum distance between text and borders along y dimension
]



\tikzfading % DO NOT CHANGE THE ORDER OF THE COLORS otherwise it will not work
[
name            = middle,
top color       = transparent!100,
bottom color    = transparent!100,
middle color    = transparent!00,
]



\tikzstyle{sCoalFiredPlant} =
[
shape           = rectangle,
minimum height  = 0.5cm,
minimum width   = 0.5cm,
rounded corners = 0.1cm,
fill            = black!20,
draw            = black,
line width      = 0.1cm,
inner xsep      = 0.2cm,
inner ysep      = 0.2cm
]


%% ===================================================
%% Add overlap and colorlet to mindmap library of tikz
%% ===================================================
\RequirePackage{helvet} 

\makeatletter 
\def\tikz@collect@child hild{% 
  \pgfutil@ifnextchar<{\tikz@collect@child@overlay}% 
  {\pgfutil@ifnextchar[{\tikz@collect@childA}{\tikz@collect@childA[]}}% 
} 
\def\tikz@collect@child@overlay<#1> 

\def\tikz@collect@child@@overlay#1[{\tikz@collect@childA[child overlay={#1},} 

\def\tikzprocessoverlay#1#2#3{% 
  \def\beamer@doifinframe{#2}% 
  \def\beamer@doifnotinframe{#3}% 
  \beamer@masterdecode{#1}% 
  \beamer@donow% 
} 

% Extra hackery to allow preactions on different layers. 
% 
\def\tikz@extra@preaction#1{% 
  {% 
    \pgfsys@beginscope% 
    \setbox\tikz@figbox=\box\voidb@x% 
    \begingroup\tikzset{#1}\expandafter\endgroup% 
    \expandafter\def\expandafter\tikz@preaction@layer\expandafter{\tikz@preaction@layer}% 
    \ifx\tikz@preaction@layer\pgfutil@empty% 
    \path[#1];% do extra path 
    \else% 
    \begin{pgfonlayer}{\tikz@preaction@layer}% 
      \path[#1];% 
    \end{pgfonlayer} 
    \fi% 
    \pgfsyssoftpath@setcurrentpath\tikz@actions@path% restore 
    \tikz@restorepathsize% 
    \pgfsys@endscope% 
  }% 
} 
\let\tikz@preaction@layer=\pgfutil@empty 

\tikzset{preaction layer/.store in=\tikz@preaction@layer} 

\makeatother 

\tikzset{% 
  child overlay/.code={% 
    \tikzprocessoverlay{#1}{}% 
    {% 
      \tikzset{% 
        circle connection bar switch color/.code={}, 
        edge from parent/.style={draw=none}, 
        every node/.style={ 
          concept, draw=none, fill=none, 
          execute at begin node={\setbox0=\hbox\bgroup\hskip0pt\let\\=\relax}, 
          execute at end node=\egroup\phantom{\box0} 
        }% 
      }% 
    }% 
  } 
} 

\pgfdeclarelayer{shadow} 
\pgfsetlayers{shadow,main} 

\tikzset{ 
  use shadow/.style={% 
    copy shadow={% 
      preaction layer=shadow, fill=gray!25, draw=none, 
      shadow xshift=0.5ex, shadow yshift=-0.5ex 
    } 
  }, 
  small mindmap/.style={ 
    level 1/.append style={level 1 concept}, 
    level 2/.append style={level 2 concept}, 
    level 3/.append style={level 3 concept}, 
    level 4/.append style={level 4 concept}, 
    every concept/.style={align=center, font=\footnotesize\strut, text=black, 
      outer sep=-.25pt}, 
    text width=2.6cm, 
    concept color=root color, 
    level 1 concept/.style={ 
      text width=2.2cm, 
      % level distance=3cm, 
      % sibling angle=75, 
      % counterclockwise from=285, 
      every child/.style={concept color=level 1 color}, 
    }, 
    level 2 concept/.style={ 
      text width=1.5cm, 
      % level distance=3cm, 
      % sibling angle=30, 
      % clockwise from=30, 
      every child/.style={concept color=level 2 color},   
    }, 
    level 3 concept/.style={ 
      text width=1cm, 
      % level distance=3cm, 
      % sibling angle=30, 
      % clockwise from=30, 
      every child/.style={concept color=level 3 color}, 
    },  
    every node/.style={concept, execute at begin node=\hskip0pt, use shadow}, 
    every circle connection bar/.append style={append after 
      command={[use shadow]}} 
  }, 
  root first/.style=, 
  root second/.style=, 
  root third/.style=, 
  root fourth/.style=, 
} 

\colorlet{root color}{blue!25} 
\colorlet{level 1 color}{purple!50} 
\colorlet{level 2 color}{red!50} 
\colorlet{level 3 color}{pink!50} 
\colorlet{highlight}{orange!25!yellow} 

\pgfkeysdefargs{/tikz/assign color}{#1 to #2}{\colorlet{#2}{#1}} 
