% --> numbered environments style ridefinition example:
%
%\newtheoremstyle{mystyle}	% name of the style
%{3pt}						% vertical space before the environment
%{8pt}						% vertical space after the environment
%{}							% body's font
%{}							% indentation (empty -> \parindent)
%{\bfseries}				% header's font
%{.}						% punctuation after the header
%{\newline}					% space after the header (empty -> \newline)
%{\thmname{#1}\thmnumber{ #2}\thmnote{ #3}}	% header formatting
%											% (cancel for the default)
%
%
% --> numbered environments ridefinition example
%
% \newtheorem	{unique_name}			% environment's ID
%										% (use \begin{unique_name})
%				[associated_counter]	% [optional] could be not
%										% unambiguous
%				{text}					% what will be printed
%				[sezione]				% [optional] to which kind of
%										% sectioning the enumeration
%										% will be tied
%
%
%
%
%
%
\newcounter{generalCounter}
\setcounter{generalCounter}{0}

\theoremstyle	{definition}
\newtheorem		{definition}	[generalCounter]	{Definition}
\newtheorem		{example}		[generalCounter]	{Example}
\newtheorem		{theorem}		[generalCounter]	{Theorem}
\newtheorem		{proposition}	[generalCounter]	{Proposition}
\newtheorem		{problem}		[generalCounter]	{Problem}
\newtheorem		{assumption}	[generalCounter]	{Assumption}
\newtheorem		{hypothesis}	[generalCounter]	{Hypothesis}
\newtheorem		{remark}		[generalCounter]	{Remark}
\newtheorem		{strategy}		[generalCounter]	{Strategy}
\newtheorem		{corollary}		[generalCounter]	{Corollary}
\newtheorem		{lemma}			[generalCounter]	{Lemma}
\newtheorem		{question}		[generalCounter]	{Question}
%
%
%
%
%
%
% usage example:
%
% \begin{theorem}[Pitagora]
% 	The square of...
% \end{theorem}



% ---------------------------------------------------------------------


% ----------------------------------------------------- % ------------------------------------------------------------
% \newcounter{nome_contatore}[contatore_gia_esistente]	% crea un contatore con il nome indicato,
% 														% il cui valore viene azzerato ogni volta che quello
% 														% opzionale (tra parentesi quadre) viene incrementato;
% ----------------------------------------------------- % ------------------------------------------------------------
% \setcounter{nome_contatore}{valore}					% assegna il valore indicato al contatore (deve
% 														% trattarsi di un valore intero, che eventualmente puo'
% 														% essere negativo);
% ----------------------------------------------------- % ------------------------------------------------------------
% \stepcounter{nome_contatore}							% incrementa il contatore di una singola unità e
% 														% azzera eventualmente i contatori che dipendono da questo;
% ----------------------------------------------------- % ------------------------------------------------------------
% \refstepcounter{nome_contatore}						% si comporta come \stepcounter, con la differenza che
% 														% coinvolge la creazione di un riferimento se seguito dal
% 														% comando \label;
% ----------------------------------------------------- % ------------------------------------------------------------
% \addtocounter{nome_contatore}{valore}					% aggiunge al contatore il valore indicato (deve
% 														% trattarsi di un valore intero, che eventualmente puo'
% 														% essere negativo);
% ----------------------------------------------------- % ------------------------------------------------------------
% \arabic{nome_contatore}								% traduce il valore del contatore in un numero arabo
% 														% nella composizione finale;
% ----------------------------------------------------- % ------------------------------------------------------------
% \thenome_contatore									% quando viene creato un contatore, si crea
% 														% implicitamente questo comando, con il quale si ottiene
% 														% il valore del contatore nella composizione finale,
% 														% espresso in modo predefinito (di solito si tratta di
% 														% un numero arabo);
% ----------------------------------------------------- % ------------------------------------------------------------
% \alph{nome_contatore}									% traduce il valore del contatore in una lettera
% 														% minuscola singola, pertanto si possono rappresentare
% 														% solo valori da 1 a 26;
% ----------------------------------------------------- % ------------------------------------------------------------
% \Alph{nome_contatore}									% traduce il valore del contatore in una lettera
% 														% maiuscola singola, pertanto si possono rappresentare
% 														% solo valori da 1 a 26;
% ----------------------------------------------------- % ------------------------------------------------------------
% \roman{nome_contatore}								% traduce il valore del contatore in un numero romano
% 														% con lettere minuscole, pertanto non si possono
% 														% rappresentare valori negativi;
% ----------------------------------------------------- % ------------------------------------------------------------
% \Roman{nome_contatore}								% traduce il valore del contatore in un numero romano
% 														% con lettere maiuscole, pertanto non si possono
% 														% rappresentare valori negativi;
% ----------------------------------------------------- % ------------------------------------------------------------
% \fnsymbol{nome_contatore}								% traduce il valore del contatore in un simbolo, ma
% 														% sono disponibili solo nove simboli, pertanto si
% 														% rappresentano valori da 1 a 9;
% ----------------------------------------------------- % ------------------------------------------------------------
% \value{nome_contatore}								% ottiene il valore del contatore, da usare all'interno
% 														% di un'espressione (non riguarda la composizione).
% ----------------------------------------------------- % ------------------------------------------------------------
