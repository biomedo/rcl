\section{Nodes}
\label{sec:nodes}




\begin{frame}
	%
	\frametitle{Nodes: how to set the shape}
	%
	(example file: \url{./Sources/nodes__examples_of_shapes.tex} - using \texttt{shapes} library)
	%
	\begin{figure}[!htb]
\begin{center}
\begin{tikzpicture}
[
	xscale	= 1,						% to scale horizontally everything but the text
	yscale	= 1,						% to scale vertically everything but the text
]

\uncover<1->
{
	\node
	(nCircle)
	[
		shape			= circle,
		top color		= white,			% | filling of the node
		bottom color	= green!20!white,	% |
		text			= red,				% colour of the fonts
		draw			= green,			% colour of the border
		rotate			= 10,				% angle of rotation
		thick,								% thickness of the border
	]
	{a circle};								% note: position not specified!
}

\uncover<2->
{
	\node
	(nStar)
	[
		right = of		  nCircle,			% position
		shape			= star,				% shape
		draw			= blue,				% colour of the border
		top color		= white,			% | filling of the node
		bottom color	= blue!20!white,	% |
		text			= blue,				% colour of the fonts
		rotate			= -10,				% angle of rotation
		thick,								% thickness of the border
	]
	{a star};
}

\uncover<3->
{
	\node
	(nStarburst)
	[
		right = of		  nStar,			% position
		shape			= starburst,		% shape
		starburst points		= 11,		% | shape parameters
		starburst point height	= 1.5cm,	% |
		draw			= red,				% colour of the border
		top color		= yellow,			% | filling of the node
		bottom color	= yellow,			% |
		text			= red,				% colour of the fonts
		font			= \Large,			% shape of the font
		rotate			= 0,				% angle of rotation
		very thick,							% thickness of the border
		drop shadow,
	]
	{a starburst};
}

\uncover<4->
{
	\node
	(nCloudCallout)
	[
		% shape and shape properties
		shape						= cloud callout,
		cloud puffs					= 11,
		aspect						= 2.5,
		cloud puff arc				= 120,
		callout pointer start size	= .25 of callout,
		callout pointer end size	= .15 of callout,
		callout relative pointer	= {(315:2cm)},	% angle - distance
		callout pointer segments	= 2,
		%
		draw			= black!90!white,	% colour of the border
		top color		= white,			% | filling of the node
		bottom color	= black!30!white,	% |
		text			= black!90!white,	% colour of the fonts
		text width		= 4cm,				%
		align			= center,			% text alignment
		very thick,							% thickness of the border
	]
	at (nStar)
	{uhm maybe better to take a look at the manual$\ldots$};
}

\end{tikzpicture}
%
\end{center}
\end{figure}


	%
\end{frame}





\begin{frame}
	%
	\frametitle{Avoid hard coding: styles definitions will make you save LOTS of time}
	%
	\begin{center}
	\begin{tikzpicture}
		%
		\node [WarningTextStyle] {declare the styles in a \\ separate file and input it somewhere};
		%
	\end{tikzpicture}
	\end{center}
	%
	(example file: \url{./Sources/graphical_settings.tex})
	%
\end{frame}




\subsection{Positioning}
\label{ssec:positioning}


\begin{frame}
	%
	\frametitle{Nodes positioning: absolute coordinates}
	%
	we can place the various nodes using absolute coordinates \\
	%
	(example file: \url{./Sources/nodes__absolute_positioning.tex})
	%
	\begin{figure}[!htb]
\begin{center}
\begin{tikzpicture}
[
	xscale	= 1,	% to scale horizontally everything but the text
	yscale	= 1,	% to scale vertically everything but the text
]


\node (nSensorOne)		[SensorNodeStyle]	at	(-0.8, 0.5)	{1};
\node (nSensorTwo)		[SensorNodeStyle]	at	(1.3, 0.9)	{2};
\node (nSensorThree)	[SensorNodeStyle]	at	(0.4, -0.9)	{3};
\node (nBayStation)		[BayStationStyle]	at	(0, 0)		{bs};


\end{tikzpicture}
\end{center}
\end{figure}


	%
\end{frame}




\begin{frame}
	%
	\frametitle{Nodes positioning: relative coordinates}
	%
	place the various nodes using relative coordinates \\
	%
	(example file: \url{./Sources/nodes__relative_positioning.tex})
	%
	\begin{figure}[!htb]
\begin{center}
\begin{tikzpicture}
[
	xscale	= 1,	% to scale horizontally everything but the text
	yscale	= 1,	% to scale vertically everything but the text
]


\node (nSensorOne)		[SensorNodeStyle]							{1};
\node (nSensorTwo)		[SensorNodeStyle, right of = nSensorOne]	{2};
\node (nSensorThree)	[SensorNodeStyle, right of = nSensorTwo]	{3};
\node (nBayStation)		[BayStationStyle, below of = nSensorTwo]	{bs};


\end{tikzpicture}
\end{center}
\end{figure}


	%
\end{frame}





\begin{frame}
	%
	\frametitle{Nodes positioning: matricial positioning}
	%
	place the various nodes inside a matrix \\
	%
	(example file: \url{./Sources/nodes__matricial_positioning.tex} - requires \texttt{matrix} library)
	%
	\begin{figure}[!htb]
\begin{center}
\begin{tikzpicture}
[
	xscale	= 1,	% to scale horizontally everything but the text
	yscale	= 1,	% to scale vertically everything but the text
]



\matrix
(mnMatrixOfNodes)	% this is the name of the ``super node''
[
	matrix of nodes,
	right delimiter	= \rmoustache,
	above delimiter	= \{,
	row sep			= 5mm,
	column sep		= 5mm
]
{
	% first row
	\node (nSensorOne) [SensorNodeStyle] {1}; & % DON'T FORGET SEMICOLONS!!
	&
	&
	\node (nSensorTwo) [SensorNodeStyle] {2}; \\ % DON'T FORGET SEMICOLONS!!
	%
	%
	% second row
	&
	&
	\node (nSensorThree) [SensorNodeStyle] {3}; & % DON'T FORGET SEMICOLONS!!
	\\
	%
	%
	% third row
	&
	\node (nSensorFour) [SensorNodeStyle] {4}; & % DON'T FORGET SEMICOLONS!!
	&
	\node (nBayStation) [BayStationStyle] {bs}; \\ % DON'T FORGET SEMICOLONS!!
};



\end{tikzpicture}
\end{center}
\end{figure}


	%
	(read the manual! This library has \alert{really useful tools}!)
	%
\end{frame}

 




\begin{frame}
	%
	\frametitle{Automatic fit of sets of nodes}
	%
	(example file: \url{./Sources/nodes__fitting_sets_of_nodes.tex} - requires \texttt{fit} library)
	%
	\begin{figure}[!htb]
\begin{center}
\begin{tikzpicture}
[
	xscale = 1.0,
	yscale = 1.0
]



\uncover<1->
{
	\node (nBayStation) [BayStationStyle] at \BayStationPosition {bs};
	\node (nSensorOne)		[SensorNodeStyle] at \SensorOnePosition			{1};
	\node (nSensorTwo)		[SensorNodeStyle] at \SensorTwoPosition			{2};
	\node (nSensorThree)	[SensorNodeStyle] at \SensorThreePosition		{3};
	\node (nSensorFour)		[SensorNodeStyle] at \SensorFourPosition		{4};
	\node (nSensorFive)		[SensorNodeStyle] at \SensorFivePosition		{5};
	\node (nSensorSix)		[SensorNodeStyle] at \SensorSixPosition			{6};
	\node (nSensorSeven)	[SensorNodeStyle] at \SensorSevenPosition		{7};
	\node (nSensorEight)	[SensorNodeStyle] at \SensorEightPosition		{8};
	\node (nSensorNine)		[SensorNodeStyle] at \SensorNinePosition		{9};
	\node (nSensorTen)		[SensorNodeStyle] at \SensorTenPosition			{10};
	\node (nSensorEleven)	[SensorNodeStyle] at \SensorElevenPosition		{11};
	\node (nSensorTwelve)	[SensorNodeStyle] at \SensorTwelvePosition		{12};
	\node (nSensorThirteen)	[SensorNodeStyle] at \SensorThirteenPosition	{13};
	\node (nSensorFourteen)	[SensorNodeStyle] at \SensorFourteenPosition	{14};
	\node (nSensorFifteen)	[SensorNodeStyle] at \SensorFifteenPosition		{15};
}




\uncover<2->
{
	\node [FittingStyle, fit=(nSensorOne) (nSensorTwo) (nSensorThree)] {};
}

\uncover<3->
{
	\node [FittingStyle, fit=(nSensorFour) (nSensorFive) (nSensorSix) (nSensorSeven)] {};
}

\uncover<4->
{
	\node [FittingStyle, fit=(nSensorSeven) (nSensorEight) (nSensorNine) (nSensorTen)] {};
}

\uncover<5->
{
	\node [FittingStyle, fit=(nSensorTen) (nSensorEleven) (nSensorTwelve)] {};
}

\uncover<6->
{
	\node [FittingStyle, fit=(nSensorTwelve) (nSensorThirteen) (nSensorFourteen) (nSensorFifteen)] {};
}




\end{tikzpicture}
\end{center}
\end{figure}
 

	%
\end{frame}



