% esempio di pagina del titolo della presentazione
%
\begin{frame}
	%
	% scrive nome, cognome, titolo della presentazione, ecc
	\titlepage
	%
	% inserisco due immagini 
	\begin{center}
	\begin{tabular}{ccc}
		%
		% immagine a sinistra
		\includegraphics[height = 1.3cm]{Logo} &
		%
		% spazio tra le immagini
		$~~~$ &
		%
		% immagine a destra
		\includegraphics[height = 1.3cm]{Logo} \\
		%
	\end{tabular}
	\end{center}
	%
\end{frame}







%%%%%%%%%%%%%%%%%%%%%%%%%%%%%%%%%%%%%%%%%%%%%%%%%%%%%%%%%%%%%%%%%%%%%%%%%%%%%%
\section{Esempi di uso del testo}


\subsection{Testo semplice}


\begin{frame}
	%
	\frametitle{Titolo della slide corrente}
	%
	\uncover<1->{Scopro dal primo click questo testo}
	%
	% lascio un po' di spazio
	\vspace*{0.4cm}
	%
	% scrivo un po' piu' in largo
	\begin{Large}
		%
		% faccio un elenco puntato (si puo' anche usare enumerate o description)
		\begin{itemize}
			%
			\item<2->{scopro questo dal \alert{secondo} click};
			%
			% un altro po' di spazio
			\vspace*{0.4cm}
			%
			\item<3->{scopro questo dal \alert{terzo} click}.
			%
		\end{itemize}
		%
	\end{Large}
	%
\end{frame}


\subsection{Testo con i boxes}


\begin{frame}
	%
	\frametitle{Titolo della slide corrente}
	%
	\begin{block}{titolo del primo blocco}
		%
		\begin{enumerate}
			%
			\item<1-> questo � un esempio
			\item<2-> di slide con un primo \ldots
			%
		\end{enumerate}
		%
	\end{block}
	%
	\begin{block}{titolo del secondo blocco}
		%
		\begin{enumerate}
			%
			\item<3-> \ldots e con un secondo blocco!
			%
		\end{enumerate}
		%
	\end{block}
	%
\end{frame}







%%%%%%%%%%%%%%%%%%%%%%%%%%%%%%%%%%%%%%%%%%%%%%%%%%%%%%%%%%%%%%%%%%%%%%%%%%%%%%
\section{Esempi di uso delle immagini}


% slide con 3 figure: una al centro e due sotto di quella al centro affiancate
% (come le ``x'' sotto):
%
%    x
%  x   x
%
\begin{frame}
	%
	\frametitle{Titolo della slide corrente}
	%
	% nota: l'overlayarea serve per allocare una zona costante di slide
	% (altrimenti il posizionamento varia da click a click)
	\begin{overlayarea}{5cm}{3cm} % {width}{height}
		%
		figura durante i clicks 1 e 2 \\
		(allineata a sinistra) \\
		%
		\begin{flushleft}
			%
			% nota: handout: il numero serve per quando si vuole stampare i lucidi
			% su carta (gli "handouts") ed indica se si vuole che la figura venga
			% stampata (1) o meno (0)
			\onslide<1- | handout:0>
			{
				\includegraphics<1-2>[height=1cm] {Logo}
			}
			%
		\end{flushleft}
		%
	\end{overlayarea}
	%
	% figure appaiate
	\begin{columns}[c]
		%
		\begin{column}{5cm}
			%
			\begin{overlayarea}{5cm}{3cm} % {width}{height}
				%
				figura durante i clicks 2 e 3 \\
				(allineata al centro) \\
				%
				\begin{center}
					%
					\onslide<2-3 | handout:1>
					{
						\includegraphics<2-3>[height=1cm] {Logo}
					}
					%
				\end{center}
				%
			\end{overlayarea}
			%
		\end{column}
		%
		\begin{column}{5cm}
			%
			\begin{overlayarea}{5cm}{3cm} % {width}{height}
				%
				figura durante i clicks 3 e 4 \\
				(allineata a destra) \\
				%
				\begin{flushright}
					%
					\onslide<3-4 | handout:0>
					{
						\includegraphics<3-4>[height=1cm] {Logo}
					}
					%
				\end{flushright}
				%
			\end{overlayarea}
			%
		\end{column}
		%
	\end{columns}
	%
\end{frame}









% slide in cui le varie figure restano sulla slide durante un numero
% limitato di click, e compaiono / scompaiono sulla stessa posizione
% (in pratica come fossero i vari fotogrammi di un ``film'')
%
\begin{frame}
	%
	\frametitle{Titolo della slide corrente}
	%
	\begin{columns}[c]
		%
		\begin{column}{5cm}
			%
			\begin{itemize}
				%
				\item<1-2> {\color[rgb]{0.6,0.6,0.6} Prima figura (slides 1-2);}
				%
				\item<3-4> {\color[rgb]{0.4,0.4,0.4} seconda figura (slides 3-4);}
				%
				\item<5-6> {\color[rgb]{0.2,0.2,0.2} terza figura (slides 5-6).}
				%
			\end{itemize}
			%
		\end{column}
		%
		%
		\begin{column}{5cm}
			%
			\begin{overlayarea}{5cm}{4cm} % {width}{height}
				%
				\begin{overprint}
					%
					% NOTE NOTE NOTE: handout:numero serve per quando si 
					%                 vuole stampare i lucidi (gli "handouts")
					%
					\onslide<1-2| handout:0>
					{
						\includegraphics<1-2>[height = 2.13cm] {first}
					}
					%
					\onslide<3-4| handout:0>
					{
						\includegraphics<3-4>[height = 2.13cm] {second}
					}
					%
					\onslide<5-6| handout:1>
					{
						\includegraphics<5-6>[height = 2.13cm] {third}
					}
					%
				\end{overprint}
				%
			\end{overlayarea}
			%
		\end{column}
		%
	\end{columns}
	%
\end{frame}






%%%%%%%%%%%%%%%%%%%%%%%%%%%%%%%%%%%%%%%%%%%%%%%%%%%%%%%%%%%%%%%%%%%%%%%%%%%%%%
\section{Conclusioni}


\begin{frame}
	%
	\frametitle{Semplice, no?}
	%
\end{frame}
